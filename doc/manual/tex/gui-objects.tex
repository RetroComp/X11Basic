\subsection{Objects}

Objects can be boxes, buttons, text, images, and more. An object tree
is an array of OBJECT structures linked to form a structured
relationship to each other. The object itself is a section of data 
which can be held by a string in X11-Basic.

The OBJECT structure is format is as follows:

\begin{verbatim}
object$=MKI$(ob_next)+MKI$(ob_head)+MKI$(ob_tail)+
        MKI$(ob_type)+MKI$(ob_flags)+MKI$(ob_state)+
        MKL$(ob_spec)+MKI$(ob_x)+MKI$(ob_y)+MKI$(ob_width)+
        MKI$(ob_height)
\end{verbatim}

An Object tree is a collection of objects:

\begin{verbatim}
tree$=object0$+object1$+ ... +objectn$
\end{verbatim}

The first object in an OBJECT tree is called the ROOT object (OBJECT 0). It's
coordinates are relative to the upper-left hand corner of the graphics window. 
The ROOT object can have any number of children and each child can have
children of their own. In each case, the OBJECT's coordinates, \verb|ob_x|,
\verb|ob_y|, \verb|ob_width|, and \verb|ob_height| are relative to that of its
parent. The X11-Basic function \verb|objc_offset()| can, however, be used to
determine the exact screen coordinates of a child object. \verb|objc_find()| is
used to determine the object at a given screen coordinate.

The \verb|ob_next|, \verb|ob_head|, and \verb|ob_tail| fields determine this
relationship between parent OBJECTs and child OBJECTs.

\begin{description}

\item[ob\_next] the index (counting objects from the first object in the
object tree) of the object's next sibling at the same level in the object tree
array. The ROOT object should set this value to -1. The last child at any given
nesting level should set this to the index of its parent.

\item[ob\_head] the index of the first child of the current object. If the
object has no children then this value should be -1. 

\item[ob\_tail] the index of the last child: the tail of the list of the
object's children in the object tree array If the object has no children then
this value should be -1.

\item[ob\_type] the object type. The low byte of the ob\_type field 
specifies the object type as follows:

\begin{center}\begin{longtable}{|cll|}
\hline
{\bf ob\_type} & {\bf Name} & {\bf Description}\\
\hline
  20 &  G\_BOX	    &   Box		\\	    
  21 &  G\_TEXT	    &   Formatted Text  \\	    
  22 &  G\_BOXTEXT   &   Formatted Text in a Box \\
  23 &  G\_IMAGE     &   Monochrome Image	\\    
  24 &  G\_PROGDEF   &   Programmer-Defined Object\\
  25 &  G\_IBOX	    &   Invisible Box		  \\  
  26 &  G\_BUTTON    &   Push Button w/String	   \\ 
  27 &  G\_BOXCHAR   &   Character in a Box	   \\ 
  28 &  G\_STRING    &   Un-formatted Text	   \\ 
  29 &  G\_FTEXT     &   Editable Formatted Text \\
  30 &  G\_FBOXTEXT  &   Editable Formatted Text in a Box\\
  31 &  G\_ICON	    &   Monochrome Icon \\
  32 &  G\_TITLE     &   Menu Title\\
  33 &  G\_CICON     &   Color Icon \\
\hline
\end{longtable}\end{center}

\item[ob\_flags] The ob\_flags field of the object structure is a bitmask of
different flags that can be applied to any object. You may want to apply 
one ore more flags at once. Just add the values ob\_flags.

\begin{center}\begin{longtable}{|clp{10cm}|}
\hline
{\bf ob\_flags} & {\bf Name} & {\bf Description}\\
\hline
0 & NONE       & No flag \\\hline
1 & SELECTABLE & object is selected. state may be toggled by clicking on it with the mouse.\\\hline
2 & DEFAULT    & An EXIT object with this bit set will have a thicker outline and be triggered when the 
                 user presses return.\\\hline
4 & EXIT       & Clicking on this OBJECT and releasing the mouse button while still over it 
                 will cause the dialog to exit.\\\hline
8 & EDITABLE   & Set for FTEXT and FBOXTEXT objects to indicate that they may receive edit focus.\\\hline
16 & RBUTTON   & This object is one of a group of radio buttons. Clicking on it will deselect
                 any selected objects at the same tree level that also have the RBUTTON flag
                 set. Likewise, it will be deselected automatically when any other object is selected.\\\hline
32 & LASTOB    & This flag signals that the current OBJECT is the last in the object tree. (Required!)\\\hline
64 & TOUCHEXIT & Setting this flag causes the OBJECT to return an exit state immediately after   
                 being clicked on with the mouse.\\\hline
256 & HIDETREE & This OBJECT and all of its children will not be drawn.\\\hline
512 & INDIRECT & This flag cause the ob\_spec field to be interpreted as a pointer to the      
                 ob\_spec value rather than the value itself.\\\hline
1024 & FL3DIND & Setting this flag causes the OBJECT to be drawn as a 3D indicator. This is  
                 appropriate for radio and toggle buttons.\\\hline
2048 & FL3DACT & Setting this flag causes the OBJECT to be drawn as a 3D activator. This is  
                 appropriate for EXIT buttons.\\\hline
3072 & FL3DBAK & If these bits are set, the object is treated as an AES background object.  
                 If it is OUTLINED, the outlined is drawn in a 3D manner. If its color
                 is set to WHITE and its fill pattern is set to 0 then the OBJECT will inherit 
                 the default 3D background color.\\\hline
4096 & SUBMENU & This bit is set on menu items which have a sub-menu attachment. This bit also
                 indicates that the high byte of the ob\_type field is being used by the menu
                 system.\\\hline
\end{longtable}\end{center}

\item[ob\_state] The ob\_state field determines the display state of the object as
follows:

\begin{center}\begin{longtable}{|clp{10cm}|}
\hline
{\bf ob\_state} & {\bf Name} & {\bf Description}\\
\hline
0 & NORMAL   & Normal state \\\hline
1 & SELECTED & The object is selected. An object with this bit set will be drawn in inverse
               video except for G\_CICON which will use its 'selected' image.\\\hline
2 & CROSSED  & An OBJECT with this bit set will be drawn over with a white cross (this   
               state can only usually be seen over a colored or	SELECTED object).\\\hline
4 & CHECKED  & An OBJECT with this bit set will be displayed with a check mark in its    
               upper-left corner.\\\hline
8 & DISABLED & An OBJECT with this bit set will ignore user input. Text objects with this   
               bit set will draw in grey or a dithered pattern.\\\hline
16 & OUTLINED & G\_BOX, G\_IBOX, G\_BOXTEXT, G\_FBOXTEXT, and G\_BOXCHAR OBJECTs   
               with this bit set will be drawn with a double border.\\\hline
32 & SHADOWED & G\_BOX, G\_IBOX, G\_BOXTEXT, G\_FBOXTEXT, and G\_BOXCHAR OBJECTs   
                will be drawn with a shadow.\\\hline
\end{longtable}\end{center}

\item[ob\_spec] The object-specific field

The ob\_spec field contains different data depending on the object type
as indicated in the table below:
\begin{center}\begin{longtable}{|lp{10cm}|}
\hline
G\_BOX &
The low 16 bits contain a WORD containing color information for the OBJECT. 
Bits 23-16 contain a signed BYTE representing the border thickness of the 
box.\\\hline
G\_TEXT &The ob\_spec field contains a pointer to a TEDINFO  structure. \\\hline  
G\_BOXTEXT &The ob\_spec field contains a pointer to a TEDINFO structure.   \\\hline
G\_IMAGE   & The ob\_spec field points to a BITBLK structure. \\\hline
G\_PROGDEF & The ob\_spec field points to a APPLBLK structure.\\\hline
G\_IBOX	  & The low 16 bits contain a WORD containing color information for the OBJECT. 
Bits 23-16 contain a signed BYTE representing the border thickness of the box.\\\hline
G\_BUTTON  & The ob\_spec field contains a pointer to the text to be contained in the button.\\\hline
G\_BOXCHAR & The low 16 bits contain a WORD containing color information for the OBJECT. 
Bits 23-16 contain a signed BYTE representing the border thickness of the box. 
Bits 31-24 contain the ASCII value of the character to display.\\\hline
G\_STRING  & The ob\_spec field contains a pointer to the text to be displayed.\\\hline
G\_FTEXT   & The ob\_spec field contains a pointer to a TEDINFO structure.\\\hline
G\_FBOXTEXT & The ob\_spec field contains a pointer to a TEDINFO structure.\\\hline
G\_ICON & The ob\_spec field contains a pointer to an ICONBLK structure. \\\hline    
G\_TITLE  & The ob\_spec field contains a pointer to the text to be used for the title.\\\hline
G\_CICON  & The ob\_spec field contains a pointer to a CICONBLK structure.\\\hline
\end{longtable}\end{center}

\begin{description}
\item[objc\_colorword]
Almost all objects reference a WORD containing the object color as
defined below.

\begin{verbatim}
objc_colorword=bbbbcccctpppcccc

Bits 15-12 contain the border color 
Bits 11-8  contain the text color 
Bit   7    is 1 if opaque or 0 if transparent 
Bits 6-4   contain the fill pattern index 
Bits 3-0   contain the fill color 
\end{verbatim}

Available colors for fill patterns, text, and borders are listed
below:
\begin{center}
\begin{tabular}{|cll|}
\hline
{\bf Value} &{\bf  Name}&{\bf  Color}  \\
\hline
 0&  WHITE	&  White\\
 1&  BLACK	&  Black\\
 2&  RED	&  Red\\
 3&  GREEN	&  Green\\
 4&  BLUE	&  Blue\\
 5&  CYAN	&  Cyan\\
 6&  YELLOW	&  Yellow\\
 7&  MAGENTA	&  Magenta\\
 8&  LWHITE	&  Light Gray\\
 9&  LBLACK	&  Dark Gray\\
 10& LRED	&  Light Red\\
 11& LGREEN	&  Light Green\\
 12& LBLUE	&  Light Blue\\
 13& LCYAN	&  Light Cyan\\
 14& LYELLOW	&  Light Yellow\\
 15& LMAGENTA	&  Light Magenta\\
\hline
\end{tabular}
\end{center}


\item[TEDINFO]
G\_TEXT, G\_BOXTEXT, G\_FTEXT, and G\_FBOXTEXT objects all reference a
TEDINFO structure in their ob\_spec field. The TEDINFO structure is
defined below:
 {\footnotesize
\begin{verbatim}                                                         
tedinfo$=MKL$(VARPTR(te_ptext$))+MKL$(VARPTR(te_ptmplt$))+
         MKL$(VARPTR(te_pvalid$))+MKI$(te_font)+MKI$(te_fontid)+
         MKI$(te_just)+MKI$(te_color)+MKI$(te_fontsize)+
         MKI$(te_thickness)+MKI$(te_txtlen)+MKI$(te_tmplen)
\end{verbatim}}

The three character pointers point to text strings required for 
\verb|G_FTEXT|
and \verb|G_FBOXTEXT| objects. te\_ptext points to the actual text to be
displayed and is the only field used by all text objects. te\_ptmplt
points to the text template for editable fields. For each character
that the user can enter, the text string should contain a tilde
character (ASCII 126). Other characters are displayed but cannot be
overwritten by the user. \verb|te_pvalid| contains validation characters for
each character the user may enter. The current acceptable validation
characters are:

\begin{tabbing}
XXXXX\=XXXXXXXXXXXX\=\kill\\
{\bf Char} \> {\bf  Allows}\\
9 \> Digits 0-9\\
A \> Uppercase letters A-Z plus space\\
a \> Upper and lowercase letters plus space\\
N \> Digits 0-9, uppercase letters A-Z and space\\
n \> Digits 0-9, upper and lowercase letters A-Z  and space\\
F \> Valid DOS filename characters plus question mark and asterisk \\
P \> Valid DOS pathname characters, backslash, colon, \\
  \> question mark, asterisk \\
p \> Valid DOS pathname characters, backslash and colon\\
X \> All characters\\
\end{tabbing}

\verb|te_font| may be set to any of the following values:

\begin{center}
\begin{tabular}{|cll|}
\hline
{\bf te\_font} &{\bf  Name} & {\bf Description} \\
\hline
3 &IBM    & Use the standard monospaced font.\\
5 &SMALL  &  Use the small monospaced font.\\
\hline
\end{tabular}
\end{center}

\verb|te_just| sets the justification of the text output as follows:

\begin{center}
\begin{tabular}{|cll|}
\hline
{\bf te\_just} &{\bf  Name} & {\bf Description} \\
\hline
0 & TE\_LEFT & Left Justify\\
1 & TE\_RIGHT & Right Justify\\
2 & TE\_CNTR & Center\\
\hline
\end{tabular}
\end{center}


te\_thickness sets the border thickness (positive and negative values
are acceptable) of the G\_BOXTEXT or G\_FBOXTEXT object. 

te\_txtlen and
te\_tmplen should be set to the length of the starting text and
template length respectively. 

\item[BITBLK]
G\_IMAGE objects contain a pointer to a BITBLK structure in their
ob\_spec field. The BITBLK structure is defined as follows:

 {\footnotesize
\begin{verbatim}                                                         
bitblk$=MKL$(VARPTR(bi_pdata$))+MKI$(bi_wb)+MKI$(bi_hl)+
        MKI$(bi_x)+MKI$(bi_y)+MKI$(bi_color)
\end{verbatim}}

\verb|bi_pdata| should contain a monochrome bit image. 
\verb|bi_wb| specifies the width (in bytes) of the image. 
All BITBLK images must be a multiple of 16 pixels wide therefore this 
value must be even.
\verb|bi_hl| specifies the height of the image in scan lines (rows). 
\verb|bi_x| and \verb|bi_y| are used as offsets into \verb|bi_pdata|. 
Any data occurring before these coordinates will be ignored. 
\verb|bi_color| is a standard color WORD
where the fill color specifies the color in which the image will be
rendered.

\item[ICONBLK]
The \verb|ob_spec| field of \verb|G_ICON| objects point to an ICONBLK structure as
defined below:

 {\footnotesize
\begin{verbatim}                                                         
iconblk$=MKL$(VARPTR(ib_pmask$))+MKL$(VARPTR(ib_pdata$))+MKL$(VARPTR(ib_ptext$))+
         MKI$(ib_char)+MKI$(ib_xchar)+MKI$(ib_ychar)+
         MKI$(ib_xicon)+MKI$(ib_yicon)+MKI$(ib_wicon)+MKI$(ib_hicon)+
	 MKI$(ib_xtext)+MKI$(ib_ytext)+MKI$(ib_wtext)+MKI$(ib_htext)
\end{verbatim}}

\verb|ib_pmask| and \verb|ib_pdata| contain the monochrome mask and 
image data respectively. \verb|ib_ptext| is a string pointer to the icon text.
\verb|ib_char| defines the icon character (used for drive icons) and the icon
foreground and background color as follows:

 {\footnotesize
\begin{verbatim}                                                         
|                              ib_char                               |
|      Bits 15-12      |      Bits 11-8       |       Bits 7-0       |
|Icon Foreground Color |Icon Background Color |ASCII Character (or 0 |
|                      |                      |  for no character).  |
\end{verbatim}}

\verb|ib_xchar| and \verb|ib_ychar| specify the location of the icon character
relative to \verb|ib_xicon| and \verb|ib_yicon|. \verb|ib_xicon| and 
\verb|ib_yicon| specify the
location of the icon relative to the \verb|ob_x| and \verb|ob_y| of the object.
\verb|ib_wicon| and \verb|ib_hicon| specify the width and height of the icon in
pixels. As with images, icons must be a multiple of 16 pixels in
width.
\verb|ib_xtext| and \verb|ib_ytext| specify the location of the text string relative
to the \verb|ob_x| and \verb|ob_y| of the object. \verb|ib_wtext| and \verb|ib_htext| specify the
width and height of the icon text area.

\item[CICONBLK]

The \verb|G_CICON| object defines its
\verb|ob_spec| field to be a pointer to a CICONBLK structure as defined
below:

 {\footnotesize
\begin{verbatim}                                                         
ciconblk$=monoblk$+MKL$(VARPTR(mainlist$))
\end{verbatim}}

\verb|monoblk| contains a monochrome icon which is rendered if a color icon
matching the display parameters cannot be found. In addition, the icon
text, character, size, and positioning data from the monochrome icon
are always used for the color one. \verb|mainlist| contains the first
CICON structure in a linked list of color icons for different
resolutions. \verb|CICON| is defined as follows:

 {\footnotesize
\begin{verbatim}                                                         
cicon$=MKI$(num_planes)+MKL$(VARPTR(col_data$))+MKL$(VARPTR(col_mask$))+
       MKL$(VARPTR(sel_data$))+MKL$(VARPTR(sel_mask$))+
       MKL$(VARPTR(cicon2$))
\end{verbatim}}

\verb|num_planes| indicates the number of bit planes this color icon
contains. \verb|col_data| and \verb|col_mask| contain the icon data and 
mask for the unselected icon respectively. Likewise, 
\verb|sel_data| and \verb|sel_mask| contain the icon data and mask for 
the selected icon. \verb|cicon2$| contains
the next color icon definition. Use \verb|MKL$(0)| if no more are available.

The GUI library searches the CICONBLK object for a color icon that has the
same number of planes in the display. If none is found, the GUI library simply
uses the monochrome icon.

\item[APPLBLK]

\verb|G_PROGDEF| objects allow programmers to define custom objects and link
them transparently in the resource. The \verb|ob_spec| field of
\verb|G_PROGDEF|
objects contains a pointer to an APPLBLK as defined below:

 {\footnotesize
\begin{verbatim}                                                         
applblk$=MKL$(SYM_ADR(#1,"function"))+MKL$(ap_parm)
\end{verbatim}}

The first is a pointer to a user-defined routine which will draw the
object. This routine must be a c-Function, which has to be linked to 
X11-basic with the LINK command. The routine will be passed a pointer to a
PARMBLK structure containing the information it needs to render the
object. The routine must be defined with stack checking off and expect
to be passed its parameter on the stack. \verb|ap_parm| is a user-defined
value which is copied into the PARMBLK structure as defined below:

{\footnotesize
\begin{verbatim}                                                         
typedef struct parm_blk {
        OBJECT          *tree;
        short            pb_obj;
        short            pb_prevstate;
        short            pb_currstate;
        short            pb_x;
        short            pb_y;
        short            pb_w;
        short            pb_h;
        short            pb_xc;
        short            pb_yc;
        short            pb_wc;
        short            pb_hc;
        long             pb_parm;
} PARMBLK;
\end{verbatim}}

\verb|tree| points to the OBJECT tree of the object being drawn. The object
is located at index \verb|pb_obj|.

The routine is passed the old \verb|ob_state| of the object in
\verb|pb_prevstate|
and the new \verb|ob_state| of the object in \verb|pb_currstate|. 
If \verb|pb_prevstate|
and \verb|pb_currstate| is equal then the object should be drawn completely,
otherwise only the drawing necessary to redraw the object from
\verb|pb_prevstate| to \verb|pb_currstate| are necessary.

\verb|pb_x|, \verb|pb_y|, \verb|pb_w|, and \verb|pb_h| give the screen 
coordinates of the object.
\verb|pb_xc|, \verb|pb_yc|, \verb|pb_wc|, and \verb|pb_hc| give the 
rectangle to clip to. \verb|pb_parm|
contains a copy of the \verb|ap_parm| value in the APPLBLK structure.
The custom routine should return a short containing any remaining
\verb|ob_state| bits you wish the GUI Library to draw over your custom object.

\end{description}

\end{description}
