
\chapter{Programming in X11-Basic}

\vspace{0.5cm}

This chapter describes all you need to know to write your own programs
in X11-Basic.

\section{The dialect of X11-BASIC}

The programming language BASIC has been around since the 1960s. BASIC is an
acronym and it stands for {\em Beginners All Purpose Symbolic Instruction
Code}. BASIC was originally designed to be a programming language that is easy
to use for a wide range of projects by anyone.
X11-Basic is a dialect of this but it is not a BASIC in its original form. It is
more a mix of classic BASIC with structured languages like PASCAL and Modula-2.
The Syntax of X11-Basic is oriented to the famous GFA-BASIC which was
developed for the ATARI ST in 1985. GFA BASIC (as of version 3.5, the most
popular one) was, by the standards of its time, a very modern programming
language. Like X11-Basic, it does without line numbers and has a
reasonable range of structured programming commands.

X11-Basic has a lot of features which make the language different from the
original (ANSI-Basic) intention. As with GFA-Basic these modifications help
developing programs with having a more structured look and which make use of
the more modern graphical user interfaces available on computers since the mid
1980's:
\begin{itemize} 
\item One command or declaration per line for better readability,
\item use of subroutines (procedures) and functions with local variables and 
      parameter passing by value or by reference,
\item data statements and arrays,
\item powerful loop and program flow constructs,
\item file and socket operations,
\item complex number mathematics,
\item operations for handling arbitrary/infinite precision numbers, 
\item commands to directly access the operation system shell,  
\item commands for using graphics in multiple windows,
\item a port of the AES (the graphical user interface 
from the ATARI ST), allowing for easy use of graphics in your program, 
\item  commands for direct memory manipulation, allowing you to access the 
computer almost as with machine language, 
\item possibility to merge source code for libraries and reuse,  
\item inline data compression and encryption (disabled in US-versions), 
\item Unicode (UTF-8) support, 
\item support to access USB and bluetooth devices,
\item powerful mathematics (including complex numbers, matrix/equation solving, 
big integers, and fast Fourier transformations), and 
\item a compiler is available.
\end{itemize}

\section*{Interpreter and Compiler}

X11-Basic programs (or scripts) are interpreted by default. This means the
so-called interpreter takes each line of your code and looks what to do with it.
The compiler does it differently, it will take your code once, translate it into
bytecode or machine code resulting in a more speedy program execution as the
step for command look-up does not appear anymore. The compiled program just can
be executed out of the box. On the other hand, the advantage of an interpreter
is that you can directly test and run your program without running a compiler
first. This is helpful while developing but of course a compiler is available
as well allowing you to produce rather fast machine code from your X11-Basic
program, once testing has been finished.

\section{Getting started}

To write a first X11-Basic program you will need an editor, where you can type
in the source code. The X11-Basic package does not include an editor, but many
so-called text editors are readily available nearly everywhere and by chance
they are already installed on your system. You can use {\it Notapad2} on MS
WINDOWS systems, {\it pico, nano, vi, emacs, nedit, gedit} and many more on
UNIX and Linux systems, {\it pico} on the TomTom device, {\it Ted} or {\it 920
text editor} on Android. This is just a small list of possibilities here. 

Open such an editor, and you can start programming.

\section{Your first X11-Basic program}

We assume, that you have opened a console window (a shell) on linux or WINDOWS. 
The Android version is a bit different. 

Open your favorite editor and type the following line of code into the editor. 

\begin{mdframed}[hidealllines=true,backgroundcolor=blue!20]
\begin{verbatim}
PRINT "Hello X11-Basic!"
\end{verbatim}
\end{mdframed}

Now save the file as "hello.bas" and run the interpreter with 
\begin{mdframed}[hidealllines=true,backgroundcolor=black!20]
\begin{verbatim}
xbasic hello.bas
\end{verbatim}
\end{mdframed}

X11-Basic should not complain. If it does, check carefully for typing mistakes.
The Program now should print out your hello message at the console or in the 
console window the interpreter was started from. It will not return to the 
shell, but just prompt for additional commands. Now type
\begin{mdframed}[hidealllines=true,backgroundcolor=black!20]
\begin{verbatim}
> quit
\end{verbatim}
\end{mdframed}
and you return to the shell. 

Of course you can include the quit command in your hello.bas:
\begin{mdframed}[hidealllines=true,backgroundcolor=blue!20]
\begin{verbatim}
PRINT "Hello X11-Basic!"
QUIT
\end{verbatim}
\end{mdframed}
Now the program always returns to the shell prompt when done. 

Now lets compile it:
\begin{mdframed}[hidealllines=true,backgroundcolor=black!20]
\begin{verbatim}
xbbc hello.bas -o hello.b
\end{verbatim}
\end{mdframed}
will produce a bytecode binary \verb|hello.b|.

You can run this:
\begin{mdframed}[hidealllines=true,backgroundcolor=black!20]
\begin{verbatim}
xbvm hello.b 
\end{verbatim}
\end{mdframed}
will give you the same output "Hello X11-Basic!".

Real compilation will need two more steps:

\begin{mdframed}[hidealllines=true,backgroundcolor=black!20]
\begin{verbatim}
xb2c hello.b -o hello.c 
\end{verbatim}
\end{mdframed}

produced a translated C-sourcefile \verb|hello.c|.

If you have the {\it gnu C compiler} available you can compile it to an 
independent executable program called hello with:

\begin{mdframed}[hidealllines=true,backgroundcolor=black!20]
\begin{verbatim}
gcc hello.c -o hello -lm -lX11 -lx11basic -lasound -lreadline
\end{verbatim}
\end{mdframed}

There you go. Your program can now directly be started with 
\begin{mdframed}[hidealllines=true,backgroundcolor=black!20]
\begin{verbatim}
./hello
\end{verbatim}
\end{mdframed}

\section{Program structure}

If you want to write more sophisticated programs than the Hello-example, you
should understand the general structure of a X11-Basic program.

A X11-Basic program consist of a main program block and subroutines. The
main program block is the shell of the program and is the section between
the first line and the keyword END (or QUIT). The code in the main block drives 
the logic of your program. In a simple program this is all that is needed. In
larger and more complex programs, putting all your code in the main block
makes the program hard to read and understand. Subroutines let you divide
your program in manageable sections, each performing its own specific, but
limited, tasks.

\section{General Syntax}

The syntax of a typical X11-Basic line is 
\begin{verbatim}
COMMAND parameters
\end{verbatim}
parameters usually consists of a list of comma separated expressions. 
Another type of X11-Basic lines are variable assignments
\begin{verbatim}
variable=expression
\end{verbatim}
variables typically have a name and can have different types. The result of the
expression will be stored under that name for further reference. 
Each line of X11-Basic code can contain exactly one command or one assignment 
(or a comment).

Here is a typical piece of X11-Basic code:

\begin{mdframed}[hidealllines=true,backgroundcolor=blue!20]
\begin{verbatim}
  LOCAL l,ll,content$,g$,gg$,comp
  CLR comp
  IF EXIST(f$)
    OPEN "I",#1,f$
    ll=LOF(#1)
    content$=INPUT$(#1,ll)
    CLOSE #1
  ENDIF
  ' and so on
\end{verbatim}
\end{mdframed}

\subsection*{Appending lines}

With many editors a limitation on the maximal line length applies (e.g. 4096
characters/line\footnote{Note, that in X11-Basic itself there is no limitation
on the line lengths.}). In X11-Basic a single command may in very rare cases
consist of more than 4096 characters (e.g. by assigning an array constant to an
array). Therefor a possibility of splitting lines into two (or more) has been
implemented. If the last character of a line is a \verb|'\'| (it must be really
the last character of the line and may not be followed by a space character!),
the following line will be appended to this line by replacing the \verb|'\'|
and the following newline character by spaces.\\ {\bf Example:} 
\begin{mdframed}[hidealllines=true,backgroundcolor=blue!20]
\begin{verbatim}
PRINT "Hello,"; \
 " that's it" 
\end{verbatim}
\end{mdframed}
 will be treated as:
\begin{mdframed}[hidealllines=true,backgroundcolor=blue!20]
\begin{verbatim} 
PRINT "Hello,";" that's it" 
\end{verbatim}
\end{mdframed}

Please note: The \verb|'\'| character must be placed at a position within the
command where a space would be allowed, too.

\subsection*{Comments}

A comment can be inserted into your program code with the REM command or the
abbreviation \verb|'|. Also the \verb|'#'| as a first character of the program
line reserves the rest of the line for a comment. Anything behind the REM will
be ignored by X11-Basic.

If you want to place comments at the end of a line, they have to be
prefaced with \verb|'!'|.\\
\begin{mdframed}[hidealllines=true,backgroundcolor=blue!20]
{\bf Example:}
\begin{verbatim}
' This is a demonstration of comments
DO    ! endless loop
LOOP  ! with nothing inside
\end{verbatim}
\end{mdframed}
Note: These end of line comments can not be used after DATA (and REM).

\section{The very BASIC commands: PRINT, INPUT, IF and GOTO}

The {\bf PRINT}-command is used to put text on the text screen. Text screen
means your terminal (under UNIX) or the console window (under Windows). PRINT
is used to generate basic output, e.g. text, strings, numbers, e.g. the result
of a calculation. Some basic formatting is possible.
\begin{mdframed}[hidealllines=true,backgroundcolor=blue!20]
{\bf Example:}
\begin{verbatim}
PRINT "The result of 1+1 is: ";1+1
\end{verbatim}
\end{mdframed}

With the {\bf INPUT} command you let the user input data, p.ex. numbers or text.
The data can be entered on the text screen/console window. Together with 
PRINT this allows already to implement a very simple user interface. 
\begin{mdframed}[hidealllines=true,backgroundcolor=blue!20]
{\bf Example:}
\begin{verbatim}
INPUT "Please enter your name: ",name$
PRINT "Hello ";name$
\end{verbatim}
\end{mdframed}

The {\bf IF} command let the program do different things depending on the result
of a calculation. Therefor the code is grouped into a block which should only be
executed if the result of the expression after IF is TRUE (this means, not
zero). The block starts with the IF command and ends with an ENDIF command. 
If the result of the expression after IF is not TRUE, means it is FALSE 
(or zero), the program will be continued after the ENDIF and the lines of 
code between the IF and the ENDIF are not run.
\begin{mdframed}[hidealllines=true,backgroundcolor=blue!20]
{\bf Example:}
\begin{verbatim}
INPUT "Please enter a number: ";a
IF a=13
  PRINT "Oh, you obviously like the thirteen!"
ENDIF
PRINT "Thank you for the ";a;"."
\end{verbatim}
\end{mdframed}

With {\bf GOTO} you can branch to a different part of your program. 
GOTO, despite its bad reputation ([goto considered harmful]), has still its 
good uses. Since no line numbers are used, you must use a label to define 
lines where the GOTO command can jump to.
\begin{mdframed}[hidealllines=true,backgroundcolor=blue!20]
{\bf Example:}
\begin{verbatim}
again:
INPUT "Please enter a number, but not the 13: ";a
IF a=13
  PRINT "Oh, you obviously like the thirteen!"
  PRINT "But, please enter a different number."
  GOTO again
ENDIF
PRINT "Thank you for the ";a;"."
\end{verbatim}
\end{mdframed}

Besides these four very basic commands (which exist in every BASIC dialect)
X11-Basic has many more features which make life easier and your programs more
user friendly.

\section{Variables}

Variables in BASIC programming are analogous to variables in mathematics. 
Variable identifiers (names) consist of alphanumeric strings. These identifiers are 
used to refer to values in computer memory.
In the X11-Basic program, a variable name is one way to bind a variable to a 
memory location; the corresponding value is stored as a data object in that 
location so that the object can be accessed and manipulated later via the 
variable's name.

\begin{mdframed}[hidealllines=true,backgroundcolor=blue!20]
{\bf Example:}
\begin{verbatim}
a=1     ! Assigns a 1 to a variable named a
b=a+1   ! The variable a can be referred to, to make a calculation
PRINT "The variable b contains now a ";b
\end{verbatim}
\end{mdframed}

Variable names can be very long if you like and also can contain digits and an
underscore, with the exception, that the first letter of the variable name must
not be a digit.
\begin{mdframed}[hidealllines=true,backgroundcolor=blue!20]
{\bf Example:}
\begin{verbatim}
my_very_long_variable_name=1.23456
PRINT my_very_long_variable_name
\end{verbatim}
\end{mdframed}

You can refer to a variable by giving its name in the place you want the value
of the variable to be used. X11-Basic will automatically know where to store the
data and how to deal with it.

It is also important to tell X11-Basic what sort of data you want to store. You
can have variables that store only numbers but also variables that deal with a
character or a whole string, a line of text for example. The following line of
X11-Basic code will create a variable called age for you and assign it the value
of 18.
\begin{mdframed}[hidealllines=true,backgroundcolor=blue!20]
\begin{verbatim}
age=18
\end{verbatim}
\end{mdframed}
But if you want to store a text, the variable needs to be capable to 
hold characters instead of numbers. Therefore, in that case you mark the 
variable with a \$ to tell that it should store text, not numbers:
\begin{mdframed}[hidealllines=true,backgroundcolor=blue!20]
\begin{verbatim}
name$="Robert"
\end{verbatim}
\end{mdframed}
Text constants, by the way, need to be enclosed with "", to tell X11-Basic that 
the text should not be interpreted as program code, but just be treated as any
text.

The assignment is done with the '=' operator. The '=' operator is also used in 
expressions, e.g. after an IF command. But there it is not used as an assignment 
operator but instead there it is treated as a comparison operator. In X11-Basic
both operators are '='. The interpreter distinguishes between them just by
context.
\begin{mdframed}[hidealllines=true,backgroundcolor=blue!20]
{\bf Example:}
\begin{verbatim}
x=(a=3)
\end{verbatim}
\end{mdframed}
Here the first = is an assignment and the second is the comparison operator.
x will be assigned a -1 (TRUE) if a is 3 and a 0 (FALSE) else. The brackets are 
not necessary here they just help reading this expression. Confused? Well, you 
eventually will get used to it.

Last to say, such an assignment will overwrite any old data that has been stored
before in that variable. As long as you don't assign a value to a variable, it
will hold a default value, 0 in most cases.

\subsection{The scope of a Variable}

X11-Basic uses two scopes for variables: global (which is the default) and 
local. 

Global variables can be modified from anywhere within the program, and
any part of the program may depend on it. Unless otherwise declared with LOCAL, 
all X11-Basic variables are global by default and this need not be explicitly 
declared.

But there is one downside of global variables: The use of global variables
makes software harder to read and understand. Since any code anywhere in the
program can change the value of the variable at any time, understanding the use
of the variable may entail understanding a large portion of the program. They
can lead to problems of naming because a global variable makes a name dangerous
to use for any other local scope variable. Also recursive programming is
nearly impossible with only global variables, last but not least, the usage of
procedures and functions becomes much more clear, if you are able to encapsulate
all internal variables of that function and you do not bother outside of the 
functions scope if you accidentally use one of these internal variables
somewhere else in the code, possibly altering the functions behavior.

Because of all this, X11-Basic also provides local variables, which live only
within a certain function or procedure and their context.

Local variables need to be declared with the command LOCAL inside the function
od procedure where they belong to. Outside this specific procedure or function 
they simply do not exist, or if a global variable of the same name exists, they 
refer to different contents.

\subsection{Data types}

Now, lets come back to the type of a variable. How can one see what kind of
content a variable can store? How does X11-Basic know? By the way the name of
the variable has been written. To distinguish between different ways of data
types X11-Basic appends a special typing sign as a suffix to the variable name
to distinguish between several ways to store data in variables. 

The X11-Basic interpreter uses 64-bit floating point variables, 32-bit integer
variables, character strings and arrays of these variables of arbitrary
dimension. A declaration of the variables and of their type is not necessary
(except for arrays $\rightarrow$ \verb|DIM|), because the interpreter recognizes
the type of the variable from the suffix: 32bit integer variables have the
suffix \%, arbitrary precision integer variables have a \&,   complex variables
a \#,  character strings a \$, arrays a (). Variables without suffix are treated
as real 64bit floating point variables. Pointers are integers,  function calls
are marked by @. Logical expressions are also of type integer. It is important
that variables with a special suffix are different from those (even if the rest
of the name is identical) without.

\begin{mdframed}[hidealllines=true,backgroundcolor=blue!20]
{\bf Examples:}
\begin{verbatim}
x=10.3        ! this is a number variable (64bit floating point)
x$="Hello"    ! this is a different character string variable
x%=5          ! this is a (32bit) integer variable, different
x&=79523612688076834923316 ! this is a big integer variable, 
                           ! still different
x#=3+4i       ! this is a complex number variable, 
@x            ! this refers to a function or procedure x
x()=[1,2,3,4] ! this beast refers to an array.
\end{verbatim}
\end{mdframed}

\subsection{Variable naming}

You can use all letters and numbers for your variable names. Spaces are not
allowed but underscores inside the variable name. The variable name can be of
any length. X11-Basic limits you only in the following ways: a variable may not
begin with a number or an underscore, only with letters. Avoid to name your
variables like X11-Basic commands. It will work but it can cause troubles. As a
rule, never try to assign values to X11-Basic system variables (like 
\verb|TRUE|, \verb|FALSE|, \verb|TIMER|, \verb|PC|, \verb|TERMINALNAME$|). The
values indeed will be assigned, but you never can use the assigned values, since
always the internal values will be used.

Valid variable names look like the following: 
\begin{verbatim}
x, auto%, lives%, bonus1%, x_1, city_name$, debit, z#   . 
\end{verbatim}

Invalid variable names look like this and X11-Basic
will complain: 
\begin{verbatim}
_blank, 1x, ?value%, 5s$, 1i, #u. 
\end{verbatim}
Always remember: begin your variable names with a letter from A-Z and you are on
the safe side!

Variable names and commands are case insensitive. Each name is bound to only
one kind of variable; \verb|A$| is a whole different variable(value) than
\verb|A| which is different from \verb|A%| or \verb|A$(1,1)|.

Space between commands will be ignored, but note that no space is allowed 
between the name of a variable or command and the '(' of its parameter list. 
So, \verb|ASC("A")| is good, \verb|ASC(     "A"     )| also,
but \verb|ASC ("A")| isn't.

\subsubsection*{Examples:}

\begin{tabbing}
XXXXXXXXXXXXXXXX\=\kill\\
integer variables: \>   \verb|i%=25| \\
		\>	\verb|my_adr%=VARPTR(b$)| \\
		\>	\verb|b%=MALLOC(100000)| \\
big integer variables: \>   \verb|i&=79523612688076834923316| \\
		\>	\verb|a&=FACT(100)| \\
float variables:  \>    \verb|a=1.2443e17| \\
		\>    \verb|b=@f(x)| \\
complex variables:  \>  \verb|a#=1.2443e17+1.2i| \\
		\>    \verb|b#=CONJ(a#)| \\
		
character strings:\>    \verb|t$="Hello everybody !"| \\
fields and arrays:\>    \verb|i%(),a(),t$(), [1,3,5;7,6,2]| \\
 \end{tabbing}

\subsection{Numbers}

X11-Basic normally uses integer numbers (32 Bit) which range from -2147483648 to
2147483647, and floating point numbers, which  are 64Bit IEEE 754 standard
values. These 64bit floating point numbers have a  mantissa of of 52 bits and an
exponent of 11 bits and a sign bit. These numbers can represent 15 to 16
significant digits and powers of 1e-308 to 1e308. Complex numbers consist of two
64bit floating point values. 

X11-Basic currently also support infinite precision integer numbers. These
numbers are stored in a variable size portion of memory, so that an arbitrary
number of digits can be stored. However, calculation with big interges is slow
and only a  few built-in functions can be used on them.

Number (constants) can be preceded by a sign, + or -, and are written as a
string of numeric digits with or without a decimal point and can also have a
positive or negative exponent as a power of 10 multiplier e.g. 

\begin{verbatim}
 -253 	67.3 	0.25 	-127.42E-3 	-1.3E7 	1
\end{verbatim}

The imaginary part of complex number constants are market with a trailling "i", e.g.

\begin{verbatim}
 -2i 	1i 	0.25+3i     -127.42E-3i
\end{verbatim}
Note: A single "i" is always treated as a real variable name. 
If you want the imaginary unit, please always use "1i".

Integer numbers, with no decimal fraction or exponent, can also be in either 
hexadecimal or binary. Hexadecimal numbers should be preceded by \$ (or 0x) and 
binary numbers preceded by \%, e.g. 

\begin{verbatim}
%101010     -$FFE0    0xA0127BD     -%10001001     %00011010
\end{verbatim}

\subsection{Strings}

String variables can contain sequences of characters (bytes) of arbitrary 
length. There is no length limit for a string other than the virtual memory of
the machine. Strings generally contain ASCII text, but can hold arbitrary byte
sequences, even characters that have the ASCII code zero. In other words a
string is a collection of bytes of certain length. You can treat strings as
arbitrary length of binary data if you need. Strings are automatically
elastic, meaning they automatically resize to contain whatever number of bytes
are put into them. When a string resizes, its location in memory may change, as
when a longer string is assigned and there is insufficient
 room after the string
to store the extra bytes.

String variables are distinguished by the \$ suffix.

String constants are enclosed with pairs of \verb|""| (double quote). 

A wealth of intrinsics and functions are provided to support efficient string
processing and manipulating. 

There is a way to include special characters into string constants. 
The usual way in BASIC is to split the string into sub strings and concatenate 
the parts during run time, like in the code fragment:
\begin{mdframed}[hidealllines=true,backgroundcolor=blue!20]
{\bf Example:}
{\footnotesize
\begin{verbatim}
st$="This is a special string, containing a bell character at the end"+CHR$(7)
\end{verbatim}
}
\end{mdframed}

By the way, the double quote character can be added with \verb|CHR$(34)|.

\subsection{Arrays}

Arrays are memory locations that store many values of the same type at the same
time. While normal variables store a single value at a time, an array variable
can store many values. The values are accessed via the name of the variable and
the appropriate indexes. The index or indexes follow the name of the variable
between ( and ). 

There is no limit on the number of indexes (the dimension). You can use as 
many as you like.
Also there is no limit on the index values other than the index values have to
be positive integer and that memory may limit the array sizes.

X11-Basic arrays can contain variables of any data type, including strings. All
arrays, even multi-dimensional arrays, can be re-dimensioned without altering
the contents. A special feature of X11-Basic is the implicit dimensioning of
arrays and the existence of array constants. You may define an array by using
the DIM command. You might also define the array by an assignment like
\begin{mdframed}[hidealllines=true,backgroundcolor=blue!20]
\begin{verbatim}
DIM b(10)
a()=b() 
\end{verbatim}
\end{mdframed}
if b() already has been DIMed or by 
\begin{mdframed}[hidealllines=true,backgroundcolor=blue!20]
\begin{verbatim}
a()=[1,2,3,4;6,7,8,9]
\end{verbatim}
\end{mdframed}
assigning an array constant. (In this example a 2 dimensional array will be
created and the rows are separated by ';'.)

\subsection{Arbitrary precision numbers}

X11-Basic also support infinite or arbitrary precision numbers with a sprecial
data type. Arbitrary-precision arithmetic, also called bignum arithmetic, 
multiple precision arithmetic, or sometimes infinite-precision arithmetic, 
indicates that calculations are performed on numbers whose digits of precision 
are limited only by the available memory of the computer. 
This contrasts with the faster fixed-precision arithmetic, normally used.

Infinite precision math is slow, and not all functions are available
for this data type. Arbitrary precision is used in applications where the 
speed of arithmetic is not a limiting factor, or where precise results with 
very large numbers are required.

The data type with the suffix \& supports (big) integers only. It is up to the 
user (and straight forward) to write routines for handling rational numbers 
(using two big interges, numerator and denominator), and corresponding routines for
adding, subtracting, multiplication and division of thouse fractions. Multiple precision 
Irrational numbers using a floiting point representation are (currently) not 
supported. If somebody needs this, please let me know.

Supported operators (for big integers) are + - * / = <> < > MOD and DIV.
Supported functions are ABS(), SQRT(), 
NEXTPRIME(), FACT(), PRIMORIAL(), FIB(),  LUCNUM(), RANDOM(), ADD(), SUB(), MUL(), DIV(), MOD(), 
POWM(), ROOT(), GDC(), LCM(), INVERT(), MIN(), MAX() and many more. 
Also STR\$(), BIN\$(), OCT\$() and HEX\$() work with big integers.

The advantage is, that you can handle big integer numbers without loosing 
precision, as it is useful for cryptography and number theory.

Also rounding errors can be avoided by using infinite-precision rational number
arithmetic (which is not implemented by X11-Basic itself, but which could be realized using pairs of
big integers.)

Variables of either type may be used and mixed in expressions. They will be converted 
to big integer or from big integer to float or 32bit integers when needed. One should j
ust be aware of the eventual loss of precision.

Here is an example how to use big number arithmetrics in X11-Basic to factorize a big
number into its prime factors:

\begin{mdframed}[hidealllines=true,backgroundcolor=blue!20]
 {\bf Example:}
\begin{verbatim}
' Factorize (big) Integer numers into prime factors.
' with X11-Basic  >= V.1.23 
'
DIM smallprimes&(1000000)
CLR anzprimes
smallprimes&(0)=2
INC anzprimes

INPUT "Enter a (big) number: ",a&
PRINT "Calculating primes up to ";lim&;". Please wait..."
lim&=SQRT(a&)   ! Limit up to which the primes are searchd
FOR i=1 TO DIM?(smallprimes&())-1
  b&=NEXTPRIME(smallprimes&(i-1))
  EXIT IF b&>lim&
  smallprimes&(i)=b&
NEXT i
anzprimes=i
PRINT "calculated ";anzprimes;" primes up to: ";b&

PRINT "Factorization:"
PRINT a&;"=";
FOR i=0 TO anzprimes-1
  WHILE (a& MOD smallprimes&(i))=0
    PRINT smallprimes&(i);"*";
    FLUSH
    a&=(a& DIV smallprimes&(i))
    lim&=SQRT(a&)
  WEND
  EXIT IF smallprimes&(i)>lim&
NEXT i
IF nextprime(a&-1)=a& or a&=1
  PRINT a&
ELSE
  ' The number is too big and we cannot be sure 
  ' that this is a prime
  PRINT "----incomplete test -----";a&
ENDIF
END
\end{verbatim}\end{mdframed}

Note that the list of small primes could also be generated by a sieve. The method  used
is based on prime number tests (using the function \verb|NEXTPRIME()|) and may be not
optimal.

\section{Arithmetics and Calculations}

X11-Basic handles numbers and arithmetic: You may calculate trigonometric
functions like \verb|SIN()| or \verb|ATAN()|, or logarithms (with \verb|LOG()|).
Bitwise operations, like \verb|AND| or \verb|OR| are available as well as
\verb|MIN()| and \verb|MAX()| (calculate the minimum or maximum of its argument)
or \verb|MOD| or \verb|INT()| (reminder of a division or integer part or a
number). Many other statements give a complete set of math functions. 

Most of these functions can work on different input data types. E.g. you can use the 
\verb|SQRT()| function also on complex numbers, thus returning a complex result. 

\subsection{Expressions and Conditions}
No difference makes the difference.
\begin{description}
\item[Expressions] are needed to calculate values. The simplest expression is a
numerical or string constant. More complex expressions may contain constants,
variables, operators, function calls and possibly parentheses. The expression
format used by X11-Basic is identical with that of many other BASIC packages:
The operators have precedence of the usual order and you can alter the order of
operator evaluation using parentheses. Here is an example numeric expression
following after a PRINT statement:
\begin{mdframed}[hidealllines=true,backgroundcolor=blue!20]
\begin{verbatim}
PRINT (x-1)*10+SIN(x)
\end{verbatim}
\end{mdframed}
\item[Conditions] and expressions are treated the same in X11-Basic. Because
X11-BASIC doesn't have separate Boolean operators for conditions and 
expressions, the operators (AND, OR, XOR, NOT) actually operate on binary
values. Therefore a TRUE is -1, which means, that every bit is one. So the
operators will operate on each of these bits. Such: a condition is considered
TRUE if the expression is not FALSE (means the result must be a value other than
zero).
\end{description}

\subsection{Operators}

X11-Basic provides operators for numerical expressions, character strings and
arrays of either type and any dimension.

\subsection*{Numerical Operators}

Numerical operators are roughly categorized in following categories:
\begin{itemize}
\item arithmetical operators:  \verb|^ * / + -| 
\item comparison operators:  \verb|= <> < > <= >=| 
\item logical operators:     \verb|NOT AND OR XOR ...|
\end{itemize}

X11-Basic recognizes the following operators, in order of falling precedence 
(the precedence of BASIC operators affects the order of expression evaluation):
\begin{center}
\begin{longtable}{|r|l|l|}
\hline
{\bf Order}&{\bf Operator}  &  {\bf Description} \\
\hline
1&\verb|( )|&  parenthetical expression\\
2&\verb|^|&    exponent/power	       \\
3&$-$	  &    sign (negation)         \\
3&$+$	  &    sign		       \\
4&NOT	  &    bitwise not	       \\
5&$/$	  &    divide 	               \\
5&*	  &    multiply 	       \\
5&DIV     &    integer division        \\
5&MOD	  &    modulus (rest of division)     \\
6&+	  &    add			      \\
6&-	  &    subtract 		      \\
7&$<<$	  &    bitwise shift to the left  (*) \\
7&$>>$	  &    bitwise shift to the right (*) \\
8&=	  &    logical "equals" 	      \\
8&<>	  &    logical "not equal"	      \\
8&<	  &    logical "less than"	      \\
8&>	  &    logical "greater than"	      \\
8&<=	  &    logical "less than or equal"   \\
8&>=	  &    logical "greater than or equal"\\
9&AND	  &    bitwise and		      \\
9&NAND	  &    bitwise not and		      \\
10&OR	  &    bitwise or		      \\
10&NOR	  &    bitwise not or		      \\
10&XOR	  &    bitwise exclusive or	      \\
10&IMP	  &    implies  		      \\
10&EQV	  &    equivalence		      \\
11& =	  &    assignment \\ 
\hline
\end{longtable}
\end{center}
(*) = not implemented

\begin{description}
\item[Addition and subtraction operators] are both binary and unary operators.
In their unary form they are used out of the precedence orders. Unary operators
are always applied first, unless parentheses drive different calculation order.
\item[The power operator] \verb|a^b| calculates the b-th power of a. The actual
implementation of the power operator always uses the \verb|pow()| function, which always
treats all operants as real numbers. Under some circumstances it might be more
optimal to use \verb|a*a| instead of \verb|a^2|.
\item[The multiplication operator] multiplies the operands. If any of the
operands is an array then the result will be an array.
\item[The division operator] divides the first operand with the second. If the
second operand is zero then an error will occur.
\item[The integer division operator] divides the first operand with the second.
The calculation is performed using integer numbers and the result is truncated
towards zero. 
\item[Bit-wise and logical NOT] This unary operator calculates the logical 
negate (the complement) of the operand. The calculation is done on integer numbers, thus the 
operand is converted to an integer value. The operator inverts each bit of the 
operand.
\item[Logical operators (AND, OR, XOR)] These operators can be used for both
logical and bit-wise operations. X11-Basic does not have a separate type for
logical values. The logical TRUE is represented as integer value -1 (all bits
set to 1) and the logical FALSE is 0. The operators AND, OR and XOR perform the
calculation on integer values. If any of the operands is not integer it is
converted to integer value before the operation takes place. The operations are
performed on each bit of the operands. 
\end{description}


\subsection*{Operators for Character Strings}
There are a few operations which can directly be done to character strings 
or string variables, using operators.
\begin{description}
\item[plus operator, conjunction] The plus '+' operator used on strings, links 
two strings together. 
\begin{mdframed}[hidealllines=true,backgroundcolor=blue!20]
{\bf Example:} 
 \begin{verbatim}
 a$="X11"
 b$="-"
 c$="BASIC"
 d$=a$+b$+c$
 \end{verbatim}\end{mdframed}
 results in a string \verb|"X11-BASIC"|.
\item[comparison operators, <, <=, =, =>, >, <>] comparison functions belong to
numerical (Boolean) functions because the result is a number, although they
can be used with strings.
\begin{mdframed}[hidealllines=true,backgroundcolor=blue!20]
{\bf Example:} 
 \begin{verbatim}
 IF a$="X11"
   ...
 ENDIF
 result=(a$<>"Hello")
 \end{verbatim}\end{mdframed}
\item[code evaluation operator, \&] the eval operator evaluates command or
expression which is given by the String. Example see below.
\end{description}

\paragraph{Rules for comparison of strings:}
\begin{enumerate}
\item	Two strings are equal if all the characters inside are identical
	(also spaces and punctuation marks).\\
 {\bf Example:}\\
	\verb|" 123 v fdh.-," = " 123 v fdh.-,"|

\item The comparison of strings with the greater and smaller operator 
works character by character until one
of them is smaller or one of the strings ends first, this is the smaller
one.\\
 {\bf Examples:}\\
\begin{verbatim}
"X11">"X11"     result:  0
"X11"<"x11"     result: -1
"123"<"abc"     result: -1
"123">"1234"    result:  0
\end{verbatim}
\end{enumerate}

\paragraph{The Evaluation Operator \&:}
The \&-operator followed by a string evaluates it for program code.\\
\begin{mdframed}[hidealllines=true,backgroundcolor=blue!20]
 {\bf Example:}
\begin{verbatim}
REM generate ten times the command 'print a$'
CLR i
a$="print a$"
label1:
INC i
IF i>10 
  b$="label2"
ELSE
  b$="label1"
ENDIF
&a$
GOTO &b$
label2:
END
\end{verbatim}\end{mdframed}
To program like this can produce a really unreadable code.


\subsection{String processing}

X11-Basic has the usual functions to extract parts from a string: 
\verb|LEFT$()|, \verb|MID$()| and \verb|RIGHT$()|. 

If you want to split a string into tokens you should use the command
\verb|SPLIT| or the function \verb|WORD$()| .

There is quite a bunch of other string-processing functions like 
\verb|UPPER$()| (converting to upper case), \verb|INSTR()| (finding one
string within the other), \verb|CHR$()| (converting an ASCII-code into a
character), \verb|GLOB()| (testing a string against a pattern) and more. 


\subsection{Arrays}
Arrays are special variables which consist of many values (of the same type).
There can be floating point arrays, integer arrays, string arrays, and arrays 
of arrays. The memory for an array need to be declared before it can be used. 
This can be done with the \verb|DIM| statement or by direcly assigning a value 
to the array.

\subsection*{Array constants}

The common way to assign data to a whole array is to put the input
figures into list into square brackets (which forms an array constant) and
assign this to an array variable like:
\begin{mdframed}[hidealllines=true,backgroundcolor=blue!20]
\begin{verbatim}
a()=[1,2,3;4,5,6]
\end{verbatim}
\end{mdframed}
A comma is used to separate columns elements, and semicolon is used to 
separate rows. So [1, 2, 3] is a row vector, and [1; 2; 3] is a column vector.

Now that you know how to define a simple array, you should know how to access 
its elements. Accessing the content of an array is done through the 
operator (), with the index inside the parenthesis; the indexing of the first 
element is 0:
\begin{mdframed}[hidealllines=true,backgroundcolor=blue!20]
\begin{verbatim}
b=a(0)
a(1)=5
\end{verbatim}
\end{mdframed}

Accessing an element outside the bounds will result in an error: 
"Field index too large."

To access a single matrix element, you can use the (i,j) subscript, where i is
the index in the row, and j in the column:
\begin{mdframed}[hidealllines=true,backgroundcolor=blue!20]
\begin{verbatim}
b=a(1,2)
a(3,4)=3
\end{verbatim}
\end{mdframed}

It is also possible to access blocks of matrices using the colon (:) operator. 
This operator is like a wildcard; it tells X11-Basic that you want all elements 
of a given dimension or with indexes between two given values. For example, 
say you want to access the entire first row of matrix a above, but not the 
second row. Then you can write:
\begin{mdframed}[hidealllines=true,backgroundcolor=blue!20]
\begin{verbatim}
b()=a(1,:)
\end{verbatim}
\end{mdframed}

Now say you only want the first two elements in the first row. To do this, use
the following syntax:
\begin{mdframed}[hidealllines=true,backgroundcolor=blue!20]
\begin{verbatim}
b()=a(1,1:2)
\end{verbatim}
\end{mdframed}

It is also possible to use arrays of any higher dimension.
\begin{mdframed}[hidealllines=true,backgroundcolor=blue!20]
\begin{verbatim}
DIM a(10,10,10,10,10)
b=a(2,5,4,2,7)
\end{verbatim}
\end{mdframed}


\subsection*{Array operators}

Arrays are not only good for storing information in tables, but one can apply
operations on arrays. You can for example use the classic arithmetic operations
+ and - on any array in X11-Basic: this results in the vector addition and
subtraction as defined in classic vector vectors spaces, which is simply the
addition and subtraction elements wise.

\begin{center}
\begin{longtable}{|c|l|}
\hline
{\bf Array Operator} & {\bf Description} \\
\hline
+ & Vector/Matrix addition element by element \\
- & Vector/Matrix subtraction element by element \\
* & Array/Matrix multiplication \\
: & Subarray (a block) \\
=,<> & comparison  element by element \\
<,>,<=,>= & comparison using a norm \\
\hline
\end{longtable}
\end{center}


Array functions and operators act on entire arrays. Some return a list, which
can then either be used as a value for another array function, or assigned into
an array variable.

Array comparisons compare the array contents element-by-element, using the
default comparison function for the element data type (=,>,<). In
multidimensional arrays the elements are visited in row-major order (last
subscript varies most rapidly). If the contents of two arrays are equal but the
dimensionality is different, the first difference in the dimensionality
information determines the sort order. 

\section{Procedures and Functions}

In X11-Basic there are two types of subroutines: procedures and functions.
The main difference between the two is that a function returns a single
value and can be used in expressions, while a procedure never returns a
value. A procedure or function must appear after the main program block.
Therefore, the structure of an X11-Basic program is as follows:

\begin{verbatim}
Main program block
END
Procedures and Functions
\end{verbatim}


\begin{description}
\item[Procedures] are blocks of code that can be called from elsewhere in a
	program. These subroutines can take arguments but return no results. They
	can access all variables available but also may have local variables
	(--> \verb|LOCAL|).

\item[Functions] are blocks of code that can be called from elsewhere within 
	an expression (e.g \verb|a=3*@myfunction(b)|). Variables are global
	unless declared local. For local variables changes outside a function
	have no effect within the function except as explicitly specified
	within the function. Functions arguments can be variables and arrays
	of any data types. Functions can return variables of any data type. By
	default, arguments are passed by value. 
\end{description}

\subsection{Procedures}

A procedure starts with the keyword \verb|PROCEDURE| followed by the procedure
name and the parameters being passed to the procedure. All procedures must
end with the keyword \verb|RETURN|. Procedures use the following format:

\begin{verbatim}
PROCEDURE ProcName(parameters)
  LOCAL vars
  procedure logic
RETURN
\end{verbatim}

The parameters of the subroutine are placed between parenthesis behind the
subroutine name and must be in the same order as the procedure call from
the main program. All variables used within the subroutine should be
declared local using the \verb|LOCAL| statement. The rest of the procedure
determines the task the subroutine must perform.

A procedure can be called in two ways: by using the keyword  \verb|GOSUB| or
\verb|@|. For instance, the procedure \verb|progress|(), which shows a progress
bar on the text console given the total amount \verb|a| and the  fraction
\verb|b|, can be  called the following ways:

\begin{mdframed}[hidealllines=true,backgroundcolor=blue!20]
\begin{verbatim}
GOSUB progress(100,i)
@progress(100,i)

PROCEDURE progress(a,b)
  LOCAL t$
  IF verbose
    PRINT CHR$(13);"[";STRING$(b/a*32,"-");">";
    PRINT STRING$((1.03-b/a)*32,"-");"| ";
    PRINT STR$(INT(b/a*100),3,3);"% ]";
    FLUSH
  ENDIF
RETURN
\end{verbatim}
\end{mdframed}


\subsection{Functions}

A function starts with a \verb|FUNCTION| header followed by a function name,
and ends with the keyword \verb|ENDFUNCTION|. The function is either a numeric
or a string function. A numeric function defaults to the floating point data
type and needs no postfix. A string function returns a string and the function
name ends with a \verb|$| postfix. A function must contain at least one
\verb|RETURN| statement to return the function value. Functions use this
format:

\begin{verbatim}
FUNCTION FuncName[$](parameters)
  LOCAL vars
  function logic
  RETURN value[$]
ENDFUNCTION
\end{verbatim}

The type of the return value must match the function type. A string function
must return a string and a numeric function a numeric value. 
The function returns to the
caller when the \verb|RETURN| statement is executed. The \verb|ENDFUNCTION|
statement only indicates the end of the function declaration and will cause an
error if the program tries to execute this statement.

A function is called by preceding the function name with \verb|@|. As an
example, the string function \verb|Copy$()| is called as follows:

\begin{mdframed}[hidealllines=true,backgroundcolor=blue!20]
\begin{verbatim}
Right$=@Copy$("X11-Basic",4)
\end{verbatim}
\end{mdframed}
where the function \verb|Copy$()| might be defined as:
\begin{mdframed}[hidealllines=true,backgroundcolor=blue!20]
\begin{verbatim}
FUNCTION Copy$(a$,p)
  LOCAL b$
  b$=MID$(a$,p)
  RETURN b$
ENDFUNC
\end{verbatim}
\end{mdframed}

Of course you are as well free to define 
\begin{mdframed}[hidealllines=true,backgroundcolor=blue!20]
\begin{verbatim}
FUNCTION Copy$(a$,p)
  RETURN MID$(a$,p)
ENDFUNC
\end{verbatim}
\end{mdframed}
instead.

An alternative for \verb|FUNCTION| is the \verb|DEFFN| statement, which 
defines a one line function.
The function \verb|Copy$()| used in the example above, might be used in a
\verb|DEFFN| statement as well:
\begin{mdframed}[hidealllines=true,backgroundcolor=blue!20]
\begin{verbatim}
DEFFN Copy$(a$,p)=MID$(a$,p)
\end{verbatim}
\end{mdframed}

In contrast with procedures and functions, \verb|DEFFN| functions may be placed
within a procedure or function body, although it doesn't use the local variables
of the subroutine. There is another difference between \verb|DEFFN| and
FUNCTION: The compiler will use the \verb|DEFFN| expression as an inline
expression and will not produce a function with a symbol name. This is a bit
faster, but produces longer code.

\subsection{Parameters and local variables}

Any X11-Basic variable type can be passed to a procedure or function. 
By default all parameters are passed "'by value"'. Though parameters can also be
passed "'by reference"' by using the \verb|VAR| statement.

The keyword \verb|VAR| precedes the list of variables that are being passed as
call by reference parameters. These variables should always be listed at
the end of the parameter list in the procedure or function heading. The
difference between the two is that a call by value parameter gets a copy of
the passed value and a call by reference does not. A \verb|VAR| variable
references the same variable that is passed to the subroutine. The original
variable will change when a subroutine modifies the corresponding \verb|VAR|
variable. In fact, both variable names reference the same piece of memory
that contains the variable value. 

Internally, X11-Basic maintains a list of all variables. Each entry in the
list points to a memory location that contains the variable value. A call
by reference variable points to the same location as the passed variable.
Therefore, constants or expressions can not be passed to a \verb|VAR| variable.

All though a (copy of an) Array can be passed to a subroutine by value, the
functions cannot return arrays\footnote{This may be possible in future versions
of X11-Basic}.

If a function needs to return information in form of an array, the return array 
sould be passed as a \verb|VAR| parameter in the parameter list. The return 
values can then be assigned to it inside the function. 

The following example shows a simple
function, which searches a name in a given string array:
\begin{mdframed}[hidealllines=true,backgroundcolor=blue!20]
\begin{verbatim}
idx%=@SearchName("Jack",Name$())

FUNCTION SearchName(n$,VAR n$())
  LOCAL idx
  CLR idx
  WHILE idx<DIM?(n$()) AND n$(idx)<>n$
    INC idx
  WEND
  RETURN idx
ENDFUNC
\end{verbatim}
\end{mdframed}

The locally used array \verb|n$()| references the global array \verb|Name$()|.
The array \verb|n$()| is only valid within the procedure, where it points to
the descriptor of the \verb|Name$()| array. 

You could as well declare the FUNCTION like 
\begin{mdframed}[hidealllines=true,backgroundcolor=blue!20]
\begin{verbatim}
FUNCTION SearchName(n$,n$())
\end{verbatim}
\end{mdframed}

Then a local copy of the whole array Name\$() would be used inside 
the function, any changes to n\$() would have no effect on the original array 
Name\$(). But win case you wanted to make changes to the array, like in following example:
 
\begin{mdframed}[hidealllines=true,backgroundcolor=blue!20]
\begin{verbatim}
idx%=@EliminateName("Jack",Name$())

FUNCTION EliminateName(n$,VAR n$())
  LOCAL i
  FOR i=0 TO DIM?(n$())
    IF n$=n$(i)
      n$(i)="deleted."
    ENDIF
  NEXT i
  RETURN i
ENDFUNC
\end{verbatim}
\end{mdframed}
you need to use VAR.

The \verb|LOCAL| statement lists the variables only known to a procedure or
function. Subroutine parameters are local variables as well.  When a
subroutine calls another subroutine the local variables of the calling routine
are known in the called routine as if they were global variables. 

Several local variables separated by commas may be listed after the
\verb|LOCAL| statement. Multiple \verb|LOCAL| lines are allowed. 


%\subsection{The address of a procedure}
%
%Some external linked shared libraries require an address of a function that can
%be called from within the shared library (or other objects of code, which can
%be linked to the X11-Basic interpreter or compiled executable, if it is
%compiled (callback functions). X11-Basic provides the \verb|_CB()=| command to
%generate a pointer to the entry point of a procedure\footnote{Not yet
%implemented to X11-Basic}.
%
%The syntax of _CB()= is as follows:
%
%_CB(n)= ProcName( [par list])
%ProcName is the place holder for an X11-BASIC procedure name, from which the
%entry point is to be determined. The parameter list specifies the number
%and size of the parameters (W for a 16 bit parameter and L for a 32-bit
%parameter.). Note, that the parameter list specifies the type of the
%variables, not the variables itself. X11-BASIC can store up to 16 (0 to 15)
%pointers to a procedure simultaneously.
%
%The following sample code shows how the address of the X11-BASIC procedure
%EnumFonts() is generated:
%
%_CB(1) = EnumFonts(l,l,w,l)
%PROCEDURE EnumFonts(plf%,ptm%,ftyp&,ldat%)
%RETURN
%X11-BASIC keeps the pointers to procedures in an internal table with 16
%entries. The _CB()= command generates and stores the address of the
%specified procedure. To read the table, you use the X11-BASIC function
%_CB(). _CB() is used to retrieve the address of the function previously
%generated with the command _CB()=. For instance, the address of the
%EnumFonts() procedure can be retrieved with:
%
%lpProc%=_CB(1)
%A memory address is always of the type long and contains a selector and
%offset value. The offset value of a starting address of a procedure is
%always zero.



\section{Simple Input/Output}

There are many ways in X11-Basic to take data by the user and display  other
data. This can be done by taking data from the keyboard, the mouse,  a
microphone, etc. and by displaying data to the text console, the  graphics
window, the speaker, etc. Also reading and writing to files and internet
connections can be done. 

The most basic input and output from and to the user is by using the text 
console, the so-called standard input and standard output. This is done in 
X11-Basic like in all BASIC dialects with the basic commands PRINT and INPUT. 

\subsection{Printing data to the console}

You actually already know a X11-Basic command to write data on screen. This
command is \verb|PRINT|. It is very versatile and you can extend it in various
ways.

The syntax of PRINT is simple: 
\begin{verbatim}
PRINT <data>
\end{verbatim}
where \verb|<data>| is whatever sort of data you want to print on screen. That
can be variables, numbers, the result of a calculation, a string or a mix of
them all. You can even add special commands and functions to your \verb|PRINT|
statement for screen control such as cursor positioning and formatting of the
data. A few examples for the \verb|PRINT| command can be found here:

\begin{mdframed}[hidealllines=true,backgroundcolor=blue!20]
\begin{verbatim}
PRINT 10+5
PRINT x%
PRINT 10;20;30
PRINT 10,20,30
PRINT "Hello!"
PRINT 10.123 USING "+##.###"
PRINT "y= ";y
PRINT "x=";x;" y=";y;" z=";z
PRINT "Your name is ";nam$
PRINT AT(5,5);"AT() is one of my favorites"
PRINT CHR$(27);"[2J This is a cleared console..."
\end{verbatim}
\end{mdframed}

These are the most simple variations of the \verb|PRINT| command. They can of 
course be more complicated and all features can be combined.

Now why do you write \verb|PRINT "y =";y| instead of \verb|PRINT "y =",y|? 
Using \verb|;| will add the following data directly behind your text without
altering the cursor position while the \verb|,| will advance the cursor to the
next vertical tabular position. You can use that to align your data in tables
on screen. In short, if you want to write data directly to some sort of prompt
or behind some text, use the \verb|;| notation. Put a \verb|;| as the last data
on your \verb|PRINT| statement to let the cursor stay on the current line. You
can use this to prevent a scrolling on the last line of the screen or if you
simply want to split writing of prompt and data into two lines of code.
Technically speaking giving the \verb|;| last will suppress a carriage return.


\subsection{Screen control}

Now that you know how to write your data on screen, you will also want to know
how to handle screen output in detail. How do I leave a line of text blank
might you ask? Write simply \verb|PRINT| without any data behind to output a
blank line on screen. Try this 3 lines program:

\begin{mdframed}[hidealllines=true,backgroundcolor=blue!20]
\begin{verbatim}
PRINT "Hello!"
PRINT
PRINT "This is the first example for screen control!"
\end{verbatim}
\end{mdframed}

As you see it prints the greeting and the other line with an empty line
between. 

A very important thing is how to clear the screen. For obvious reasons, you'll
sometimes prepare a screen layout that requires you not to have other text or
old data on screen. You'll simply clear the screen with the following command.

\begin{mdframed}[hidealllines=true,backgroundcolor=blue!20]
\begin{verbatim}
CLS
\end{verbatim}
\end{mdframed}

A neat thing is to write on screen exactly on a position where you want and not
following the listed flow of ordinary \verb|PRINT| statements. You can use the
\verb|AT()| statement. This special addition for \verb|PRINT| allows you to
position the cursor freely on screen so you can write your data where you want.
Let's try the following example program:

\begin{mdframed}[hidealllines=true,backgroundcolor=blue!20]
\begin{verbatim}
CLS
PRINT AT(1,1);"Top left"
PRINT AT(5,13);"Middle line, text indented 5 chars"
PRINT AT(20,25);"bottom line";
\end{verbatim}
\end{mdframed}

The syntax for \verb|PRINT AT();| is \verb|PRINT AT(column, row);|, where row 1 is
on top of the screen and column 1 on the left end. Column and row can be
variables, expressions or simply a plain number. Valid \verb|PRINT AT()|
commands are:

\begin{mdframed}[hidealllines=true,backgroundcolor=blue!20]
\begin{verbatim}
PRINT AT(1,5);"Hello"
PRINT AT(5+x%,10);"x"
PRINT AT(4+8,y%);"y = "
\end{verbatim}
\end{mdframed}

How many character positions you have depends on the current text console screen
size. You have almost always at least 24 lines of text. 80 columns are standard.
If you want to exactly know the number of rows and columns of the text screen,
you can use the (system) variables \verb|ROWS| and \verb|COLS|.

\begin{mdframed}[hidealllines=true,backgroundcolor=black!20]
\begin{verbatim}
> PRINT ROWS,COLS
24      80
\end{verbatim}
\end{mdframed}

There are more commands you can use with \verb|PRINT| like \verb|SPC()| and
\verb|TAB()|. Refer to the command reference on them.

\subsection{Formatting output with PRINT USING}

X11-BASIC normally prints numbers in a form convenient for most purposes. But on
occasion you may prefer a more elaborate form. For example, you may want to
print financial quantities with two decimal places (for cents) and, possibly,
with commas inserted every three digits to the left of the decimal point. Or, 
you want to print the numbers in scientific notation. 
\verb|PRINT USING| provides a way to print numbers in this and almost any other
form\footnote{There are also other built-in commands for formatting data on
output. X11-Basic offers e.g. STR\$(). Please refer to the sections on
functions for details on the syntax of STR\$().}.

The generic syntax is 
\begin{verbatim}
PRINT <expression> USING "<format string>"
\end{verbatim}. The
result of the expression should be a number. The format string defines how you
want your data to be formatted on screen. The format string may be a string
variable, a quoted string, or a more general string expression.

\verb|PRINT USING| also allows one to print strings centered or right-justified, as
well as left justified.

The function \verb|USING$()| duplicates the PRINT USING statement almost 
exactly\footnote{only numbers can be formatted, no strings.} but returns the 
result as a string rather than printing it on the screen. 

Unlike \verb|STR$()|, where you can specify the length of the string, the
number of significant digits of the number and a flag, where there leading zeroes
should be used, \verb|USING$()| and \verb|PRINT USING| use a classic and
BASIC-style formatter string for formatting numbers. The difference between
\verb|USING$()| and \verb|PRINT USING| is just, that \verb|PRINT USING|
immediately prints out the formatted number and \verb|USING$()| converters it
into a string containing the formatted number, suitable for further processing.

\subsubsection*{Formatting Numbers}

The format string can contain any letters, but some have a special meaning. 
All other characters are just taken as they are. The length of the format 
string defines the length of the output field. Whatever is formatted, it 
will exactly take as many characters as the length of the format string. 

The most important special character in the format string is the symbol \#,
which stands for a digit position to be filled with one digit from the number
to be formatted. For example, compare the output resulting from two similar
PRINT statements, the first a normal PRINT statement and the second employing
USING.

\begin{verbatim}
x=   |PRINT x| PRINT x USING "###"
-----+-------+--------------------
1    | 1     |   1
12   | 12    |  12
123  | 123   | 123
1234 | 1234  | ***
-12  | -12   | -12
\end{verbatim}

Without USING, the number is printed left justified and occupying only as much 
space as needed.
With USING, the format string \verb|"###"| specifies a field length of exactly
three characters. The number is printed right-justified in this field. If the
field is not long enough to print the number properly, asterisks are printed
instead. If all you need to do is to print integer numbers in a column but with
right-justification, then the preceding example will suffice. Note that a 
negative number will be printed with the sign occupying one of the digit fields.

With printing financial quantities it is conventional that the decimal points 
are aligned. Also, you may want to print two decimal places (for the cents) even
when they are zero. The following
 example shows how to do this. (In
order to print negative numbers and have the sign at a fixed position, the 
format string should start with a minus sign.)

\begin{verbatim}
x=    |PRINT x USING "-##.##"
------+-----------------------
1     |  1.00
1.9   |  1.90
-3.14 |- 3.14
1.238 |  1.24
123   |******
0     |  0.00
-123  |******
\end{verbatim}

Notice that in this example two decimal digits are always printed, even when
they consist of zeroes. Also, the result is first rounded to two decimals. If
the number is negative, the minus sign occupies the leading digit position or
the position given by a - or + in the format string. If the number is too long
to be printed properly (possibly because of a minus sign), asterisks are printed
instead.

Financial quantities are often printed with a leading dollar sign (\$), and 
with commas forming three-digit groups to the left of the decimal point. The 
following example shows how to do this with \verb|PRINT USING|.

\begin{verbatim}
x=         |PRINT x USING "$#,###,###.##"
-----------+------------------------------
0          |$        0.00
1          |$        1.00
1234       |$    1,234.00
1234567.89 |$1,234,567.89
1e6        |$1,000,000.00
1e7        |10,000,000.00
1e8        |*************
\end{verbatim}

The dollar sign is only printed if the space is not needed for a digit. It is
always in the same position (first) in the field. The separating commas are
printed only when needed.

In case you want the dollar sign (\$) to float to the right, so that 
it appears next to the number, avoiding all those blank spaces between the 
dollar sign and the first digit in the preceding example. The following 
example shows how to do this.

\begin{verbatim}
x=         |PRINT x USING "$$,$$$,$$#.##"
-----------+----------------------------
0          |        $0.00
1          |        $1.00
1234       |    $1,234.00
1234567.89 |$1,234,567.89
\end{verbatim}

The format string can also allow leading zeroes to be printed, or to be 
replaced by asterisks (*). You might find the latter useful if you are 
preparing a check-writing program.

\begin{verbatim}
x=         |PRINT x USING "$0,000,000.##"
-----------+------------------------------
0          |$0,000,000.00
1          |$0,000,001.00
1234       |$0,001,234.00
1234567.89 |$1,234,567.89

x=         |PRINT x USING "$*,***,***.##"
-----------+------------------------------
0          |$********0.00
1          |$********1.00
1234       |$****1,234.00
1234567.89 |$1,234,567.89

x=         |PRINT x USING "*$$,$$$,$$#.##"
-----------+------------------------------
0          |*********$0.00
1          |*********$1.00
1234       |*****$1,234.00
1234567.89 |*$1,234,567.89
\end{verbatim}

For compatibility reasons, a \% can be used instead of the 0's in the 
format string, with one exception: The first character in the format 
string must not be a \%\footnote{If the first character is a \% the 
format string is interpreted as a C style printf format string (see below).}. 

You can also format numbers using scientific notation. Because scientific 
notation has two parts, the decimal-part and the exponent-part, the format 
string must also have two parts. The decimal-part follows the rules already 
illustrated. The exponent-part consists of from three to five carets (\^) that 
should immediately follow the decimal-part. The following example shows how.

\begin{verbatim}
x=          |PRINT x USING "+#.#####^^^^"
------------+-----------------------------
0           |+0.00000e+00
123.456     |+1.23456e+02
-.001324379 |-1.32438e-03
7e30        |+7.00000e+30
0.5e100     |+5.00000e+99
5e100       |************
\end{verbatim}

The leading plus sign (+) in the format string guarantees that the
sign of the number will be printed, even when the number is positive. Notice
that the last number cannot be formatted because the exponent part would
have been 100, which requires an exponent field of five carets. Notice also that
if there are more carets than needed for the exponent, leading zeroes are
inserted. Finally, notice that trailing zeroes in the decimal part are printed.

In addition to the format rules explained above, X11-Basic offers another but
different set of format strings. If the first character of the format string 
is a \% the format string is treated as a C style so-called printf-formatter.

Here are some examples:

\begin{verbatim}
x=          |format$=     |PRINT x USING format$
------------+------------------------------------
0           | "%012g"     |000000000000
123.456     | "%.1g"      |1e+02
-.001624    | "%.1g"      |-0.002
\end{verbatim}

These formatting strings follow some standard which is normally not used in
BASIC. The standard is well explained in Wikipedia: 
\url{http://en.wikipedia.org/wiki/Printf_format_string#Format_placeholders}.

\subsubsection*{Formatting Strings}

Strings can also be formatted through \verb|PRINT USING| but not with the 
function \verb|USING$()|, although there are fewer options for strings than 
for numbers. Strings can be printed in the formatted field either 
left justified, centered, or right-justified. 
As with numbers, if the string is too long to fit, then asterisks are printed. 

These examples should make it clear:
\begin{mdframed}[hidealllines=true,backgroundcolor=blue!20]
\begin{verbatim}
PRINT "|";"OK" USING "#####";"|"     ! result: | OK  |
PRINT "|";"OK" USING ">####";"|"     ! result: |   OK|
PRINT "|";"Hello" USING ">####";"|"  ! result: |Hello|
PRINT "|";"Goodby" USING ">####";"|" ! result: |*****|
\end{verbatim}
\end{mdframed}

If centering cannot be exact, the extra space is placed to the right. Actually
any string can be used as a format string. Only the length of the string
defines the length of the output field. Only the first character of the format
string matters. If it is a < the string will be left justified, if it is a > it
will be right-justified and centered in any other case. This is especially
valuable for printing headers for a numeric table. The following example shows
how you can format headers using the same format string we used earlier for
numbers.

\begin{verbatim}
s$=                   |PRINT s$ USING "$#,###,###.##"
----------------------+-------------------------------
"Cash"                |                    Cash
"Liabilities"         |                 Liabilities
"Accounts Receivable" |                *************
\end{verbatim}

\subsection{Gathering User Input}

You can make your program interactive and ask the user to enter data on runtime
of your program.

The command \verb|INPUT| allows the user to enter one line of data with the 
keyboard on the text console. The data is interpreted and stored in one or more
variables specified by the INPUT statement. If you specify a string variable,
you can enter text while you can only enter numeric data if you use a numeric
variable. A minus sign and optional decimal point are allowed for numeric input.
Also numbers can be entered in scientific notation. Hexadecimal values are
possible, too.

\begin{mdframed}[hidealllines=true,backgroundcolor=blue!20]
\begin{verbatim}
INPUT "x= ",x
INPUT "What is your name? ",your_name$
\end{verbatim}
\end{mdframed}

This will prompt the user to enter a value for x which will be stored into a
(floating point) variable x. You can then use this variable in your program as
normal, doing calculations with it. Please note that your program will stop 
until the RETURN key or the ENTER key has been pressed to terminate the input.

You can read more than one variable with one \verb|INPUT| statement, just list
your variables where you want your input to go to with separating commas.

\begin{mdframed}[hidealllines=true,backgroundcolor=blue!20]
\begin{verbatim}
PRINT "enter 3 values, separated with commas (eq 3,4,5):"
INPUT x%,y%,z%
\end{verbatim}
\end{mdframed}

The user has then to enter commas at the appropriate places to tell which
input goes to which variable. To the example above the user would respond with
5,6,7. 

\begin{mdframed}[hidealllines=true,backgroundcolor=blue!20]
\begin{verbatim}
CLS
INPUT "Enter a value for x:",x
PRINT "x = ";x
INPUT "What is your name?",your_name$
PRINT "Your name is ";your_name$;"."
PRINT "Bye, ";your_name$;"!" 
\end{verbatim}
\end{mdframed}

While entering strings you may have already noticed that X11-Basic will treat
entering a comma again as a delimiter, effectively cutting your string at that
comma. Use the command \verb|LINEINPUT| instead of \verb|INPUT| to read strings.

\begin{mdframed}[hidealllines=true,backgroundcolor=blue!20]
\begin{verbatim}
LINEINPUT txt$
\end{verbatim}
\end{mdframed}

You can now enter strings with a comma in and it will be saved to the string
variable as well. You can read multiple strings with \verb|LINEINPUT| as well but the
user has to press the RETURN key terminating each string to be entered.

\section{Flow Control}

This time you'll finally make your programs do things more then once without
having to retype your code. The creation of so-called loops is essential for
making complex programs work. The concept of looping and simple counting loops

Before going further let me explain you the fundamental idea of looping. The
idea is to make your program repeat a section of code for a defined amount of
time. You may let X11-Basic count a variable for you and you can then use the
value of that variable in an ongoing calculation. Or you can let X11-Basic loop
a certain part of code until a special condition has been met. Take a look at
the following sample program:

\begin{mdframed}[hidealllines=true,backgroundcolor=blue!20]
\begin{verbatim}
FOR i%=1 TO 5
  PRINT i%
NEXT i%
\end{verbatim}
\end{mdframed}

This little example program loops 5 times and counts the variable i\% from 1 to
5 and prints the current value to the screen. This sort of loop is called a
FOR-NEXT-loop. You can use any numerical variable to count. Most often this
sort of loop is used to do things a certain amount of time or to iterate over a
list. The loop will repeat the code between the \verb|FOR| and its corresponding NEXT.
Each time X11-Basic reaches the \verb|NEXT|, it will increment the count variable and
will stop the loop if the maximum count has been reached.

You can of course have another loop inside the current one. Just make sure not
to use the same variable for counting or X11-Basic will do unpredictable things:
\begin{mdframed}[hidealllines=true,backgroundcolor=blue!20]
\begin{verbatim}
FOR i%=1 TO 5
  FOR j%=1 TO 10
    PRINT i%;" * ";j%;" = ";i%*j%
  NEXT j%
NEXT i%
\end{verbatim}
\end{mdframed}

That sample program has one FOR-NEXT-loop in another and it calculates the
product of the both counter variables creating some sort of multiplication
table. Some rules and advice to keep in mind with FOR-NEXT-loops:

\begin{enumerate}
\item Always terminate an opened \verb|FOR| with a corresponding \verb|NEXT|.
\item Always
terminate FOR-loops in the correct order. If you write \verb|FOR i%=|\dots first and 
\verb|FOR j%=|\dots next, make sure to terminate the inner loop first.
\item You can count
downwards with the word \verb|DOWNTO| instead of \verb|TO|.
Try:
\begin{mdframed}[hidealllines=true,backgroundcolor=blue!20]
\begin{verbatim}
FOR i%=5 DOWNTO 1
\end{verbatim}
\end{mdframed}
\item You can count in steps not equal 1 with the keyword \verb|STEP|: 
\begin{mdframed}[hidealllines=true,backgroundcolor=blue!20]
\begin{verbatim}
FOR i%=1 TO 10 STEP 2
\end{verbatim}
\end{mdframed}
That will increment i\% in steps of 2 until it reaches 10. 
\item X11-Basic will
check for correct loop termination while entering the code into the editor. 
\item You can terminate the FOR-NEXT-loop with the \verb|EXIT IF| statement.
\end{enumerate}

\subsubsection{Conditions}

A very fundamental idea in programming is to create and use conditionals. These
will allow you to make decisions when certain conditions are met and let your
program take an alternative code segment.

Try to imagine that you count a special variable and want to do something else
when the value of your counter is 5:
\begin{mdframed}[hidealllines=true,backgroundcolor=blue!20]
\begin{verbatim}
FOR i%=1 to 10
  IF i%=5
    PRINT "i% is now 5"
  ELSE
    PRINT "i% is not 5"
  ENDIF
NEXT i%
\end{verbatim}
\end{mdframed}

This program loops 10 times and counts in the variable i\%.
For each iteration of the loop it checks if i\% is 5 in the IF line. If that
condition is true, i\% is 5, then it executes the program branch until the ELSE
and omits the following part. If the condition is not true, X11-Basic will only
execute the part behind the ELSE. Make sure to terminate each \verb|IF| conditional
with an \verb|ENDIF| or X11-Basic will get lost and produce an error message. 

You may leave out the \verb|ELSE| fork. X11-Basic will then do nothing if the 
condition is not true.

\subsection{Conditional and endless loops}


Sometimes you don't know how far you need to count for a special operation. Or
imagine a game. You don't want to let it run just for 10 frames but until the
player sprite did collide or something like that. The first new loop will loop
until a condition is fulfilled:

\begin{verbatim}
REPEAT
...
UNTIL <condition>
\end{verbatim}

This is a so-called REPEAT-UNTIL-loop. It loops at least once and checks for
the condition after the loop contents have been executed by X11-Basic. Use it
for things that need to be done at least once. You can emulate FOR-NEXT-loops
with it if you want trickier counting:
\begin{mdframed}[hidealllines=true,backgroundcolor=blue!20]
\begin{verbatim}
i%=1
REPEAT
  PRINT "i%=";i%
  i%=i%+1
UNTIL i%>5
\end{verbatim}
\end{mdframed}

Surely you can test the condition before entering a loop. This is useful if you
want to loop only when a certain condition is already true:

\begin{verbatim}
WHILE <condition>
...
WEND
\end{verbatim}

This is the so-called WHILE-WEND loop. It checks the condition first and it
will not execute the loop body if the condition is not fulfilled. Sometimes you
want to loop endless. X11-Basic has a special loop construct for this purpose
although you can create never ending loops easily with the types above if you
use a condition that will never get true. The never ending loop is called
DO-loop. The 3 loops in the example are all equal in functionality and will
loop endless.

\begin{mdframed}[hidealllines=true,backgroundcolor=blue!20]
\begin{verbatim}
DO
  PRINT "endless"
LOOP

i%=0
REPEAT
  PRINT "endless"
UNTIL i%=1

i%=0
WHILE i%=0
  PRINT "endless"
WEND
\end{verbatim}
\end{mdframed}

At this point it is important that you know you can terminate at your X11-Basic
program at any point. This is useful if your program
gets stuck in an endless loop which was not intended. Press CONTROL-c together 
and X11-Basic will stop the program. Another CONTROL-c will quit the interpreter.

Sometimes you will want to terminate a running loop at another point than the
official loop beginning or loop end. Use the \verb|EXIT IF| statement in your loop for
extra conditions. This will also terminate FOR-NEXT-loops if you wish to and it
is the only way to terminate a DO-LOOP.

\begin{mdframed}[hidealllines=true,backgroundcolor=blue!20]
\begin{verbatim}
i%=1
DO
  PRINT "i%=";i%
  EXIT IF i%=5
  i%=i%+1
LOOP
\end{verbatim}
\end{mdframed}

Please note that the \verb|EXIT IF| statement has no \verb|ENDIF| or the like.
It just terminates the loop and continues your program behind the loop end.

\section{Diagnostics}

Some errors can be caught by the user program using the \verb|ON ERROR GOTO| or
\verb|ON ERROR GOSUB| command. If no error trapping routine has been supplied
then program execution is terminated and a message is printed with the
corresponding line number. The error messages are standardized. Each error 
message has a corresponding number which can be referenced by the system variable 
\verb|ERR|. A list of standard error messages by number can be found in chapter
\ref{errors}.
