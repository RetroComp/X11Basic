\chapter{About X11-Basic}

X11-Basic is a dialect of the BASIC programming language with graphics and sound
which integrates features like traditional BASIC language syntax, structured
programming, shell scripting, cgi programming, powerful math, and full graphical
visualization into the easy to learn BASIC language on modern  computers.

The syntax of X11-Basic is most similar to GFA-Basic in its original ancient 
implementation for the ATARI ST. Old GFA-programs should run with only a few 
changes. Also DOS/\-QBASIC programmers will feel comfortable.

X11-Basic is as well suited to novices as programming wizards, and is
appropriate for virtually all programming tasks. For science and engineering
X11-Basic has already proven its capability of handling complex simulation
and control problems. For system programs, X11-Basic has high level language
replacements for low level programming features that are much easier to read,
understand, and maintain. For all applications, X11-Basic is designed to support
rapid development of compact, efficient, reliable, readable, portable, well
structured programs.

X11-Basic supports the principle 'small is beautiful'. Its aim is to use the 
fewest system resources and execute with the highest speed. X11-Basic meets in
this, by providing very powerful built-in commands and functions, and a very fast
compiler producing even faster applications. X11-Basic lets you write an
application with very little effort, giving you full control over your
application. X11-Basic doesn't use "black boxes" with an enormous overhead, but
instead calls operating system functions whenever possible. In case the
X11-Basic commands and functions aren't sufficient, you can easily use the
native shell to execute other programs and commands, or you will be able to use
any shared library on the system, which can be dynamically linked.

No language is perfect and X11-Basic is no exception. It has its weak and
it's strong points. You won't use X11-Basic to write major applications, but
it is extremely well suited to develop small to medium sized programs.
X11-Basic is nearly as versatile as C, it uses procedures and functions and
parameter passing similar to C.

X11-Basic programs are constructed in a straightforward fashion. As C, X11-Basic
doesn't use object oriented structures and allows an easy start. 

Because it is an interpretive language each new step in your program can be
tested quickly providing you with instant feedback. And when you finished your
program you can use the X11-Basic compiler to create a very fast stand-alone
executable. No complicated compiler options and linker switches are necessary to
create a stand-alone application.

\section*{Portability}

X11-Basic is designed to run on many platforms with extremely low resources.  It
has started on UNIX workstations and Linux-systems with the X-Window system 
(commonly known as X11, based on its current major version being 11).  In case
where no X11 implementation is available, X11-Basic can be compiled with a
framebuffer-device graphics engine. The Android version e.g. uses the
framebuffer interface. Also such a version for the TomTom navigation devices has
been created.

Porting X11-Basic to more basic and embedded systems with a very low amount of
RAM and processing speed is well possible. On UNIX and Linux systems, not  only
the X11 graphics engine can be used, but also the SDL library  (=Simple
Direct-Media Library), as well as any raw framebuffer device or no graphics at
all. The MS WINDOWS version supports only SDL (or no graphics at all).

X11-Basic supports complex numbers and complex math, as well as arbitrary 
precision numbers and calculations where needed, as well as very fast 32bit 
integer and 64bit floating point operations, very powerful string handling 
functions for charackter strings of any length and any content. 

Sound is not available on every system. Where available, X11-Basic implements a 
16 channel sound synthesizer as well as the option to play sound samples from
standard sound file formats (line .wav and .ogg). On LINUX systems the ALSA
sound engine is used. The Android port of X11-Basic uses the Android sound and
speech engine.

The X11-Basic environment contains a library of GEM\footnote{GEM=Graphics
Environment Manager, an operating environment created by Digital Research, Inc.
(DRI), which was used on the ATARI ST and GFA-BASIC.} GUI\footnote{GUI=Graphical
User Interface} functions.  This makes writing GUI programs in X11-Basic faster,
easier and more portable than programming with native GUI tools.

The Android version of X11-Basic contains a full featured coloured 
VT100/ANSI terminal emulation and support for unicode character sets 
(UTF-8 coded) for standard output.


\section*{Structured programming}

X11-Basic is a structured procedural programming language.  Structure is a form
of visual and functional encapsulation in which multiple-line sections of
program look and act like single units. The beginning and end of blocks are
marked by descriptive keyword delimiters.

In contrast to more traditional BASIC implementations, line  numbers are not
used in X11-Basic. Every line holds only one instruction. Jumps with GOTO are
possible but not necessary. All the well-known loops are available including 
additional commands for discontinuation ($\longrightarrow$ \verb|EXIT IF|, \verb|BREAK|). 

Procedures and functions with return values of any type can be defined. This way
BASIC  programs can be structured in a modular way. A program can contain a main
part to call subfunctions and subprocedures, which may or may not be defined in
the same source file. Distinct sources can form a library. Whole libraries can
be added with the merge command ($\longrightarrow$ \verb|MERGE|).

To help porting ANSI-Basic\footnote{So-called ANSI-Basic has been  standardized
by the American National Standards Institute. ANSI-Basic uses line numbers and
the syntax can be quite different from X11-Basic.} programs (with line numbers)
to X11-Basic, a converter ($\longrightarrow$ \verb|bas2x11basic|) has been written. It comes
with the X11-Basic package. 

The third-party tool \verb|gfalist|\footnote{You will find a link to gfalist 
(the project name is ONS) on the X11-Basic homepage.} by Peter Backes (not
included in the X11-Basic package) even allows to decode GFA-Basic \verb|.gfa|
files to ASCII.

\section*{Speed of X11-Basic}

How fast is X11-Basic? The answer depends on the way an X11-Basic program is
run: It depends on if the code is interpreted, run as bytecode in a virtual
machine, or being compiled to native machine language. Generally we find:

\begin{enumerate}
\item X11-Basic programs run by the interpreter are slow,
\item X11-Basic programs compiled to bytecode and then run in the X11-Basic 
virtual machine (\verb|xbvm|) is fast, but
\item X11-Basic bytecode compiled natively to real machine language 
is even faster.
\item arbitrary precision numbers and calculations are slow, but 
\item 64bit floating point and complex number calculations as well as 
32bit integers are very fast.
\end{enumerate}

Bytecoded programs are always interpreted faster than scripted programming 
languages. The X11-Basic compiler can translate the X11-Basic bytecode to C, 
which then can be compiled to native machine language using any C-compiler 
(preferably \verb|gcc| on UNIX systems). Obviously your programs will be slower
than optimized C/C++ code but it already comes close.

If you need highest possible speed you can load and link a separate
DLL/shared object with the time critical part of your code written in another 
language (e.g. C or Assembler). 

A speed comparison was done with the Whetstone benchmark ($\longrightarrow$
\verb|Whets.bas|).  This shows, that bytecode-programs are about 19 times faster
than the interpreted code and a natively compiled program can run about 28 times
faster.

\section*{Optimality of code and code overhead}

At a minimum the X11-Basic interpreter and the bytecode interpreter  (virtual
machine) require about 350~KB of memory and another 400~kB of file  size, which
includes the X11-Basic runtime-library.  So this is the overhead that all your
programs will have. Compared to some Windows programs, this isn't that bad. Most
likely your bytecode is less than 50~kB anyway (for a moderate/large
application), plus any resources and graphics you may want to include of course.
In the end the code  produced will be reasonably small and light enough to be
also used on  portable devices (e.g. cell phones, e-book readers, and navigation
devices) which have only  a small amount of native memory (and a relatively slow
processor).

\section*{Copyright information}

Copyright (C) 1997-2015 by Markus Hoffmann 

Permission is granted to copy, distribute and/or modify this document
under the terms of the GNU Free Documentation License, Version 1.2
or any later version published by the Free Software Foundation;
with no Invariant Sections, no Front-Cover Texts, and no Back-Cover Texts.
A copy of the license is included in the section entitled "GNU
Free Documentation License".

X11-Basic is free software; you can redistribute it and/or modify it under the
terms of the GNU General Public License as published by the Free Software
Foundation; either version 2 of the License, or (at your option) any later
version.

This program is distributed in the hope that it will be useful, but WITHOUT ANY
WARRANTY; without even the implied warranty of MERCHANTABILITY or FITNESS FOR
A PARTICULAR PURPOSE. See the GNU General Public License for more details.

Read the file COPYING for details.

(Basically that means, free, open source, use and modify as you like, don't
incorporate it into non-free software, no warranty of any sort, don't blame me
if it doesn't work.)
